\documentclass[12pt,oneside,listof=totoc,paper=a4,headings=small]{scrbook}
% Nützliche Packages für die Gestaltung und allgemeine Konfiguration des Dokuments
% -----------------------------------

% Allgemeine Formatierungen
\usepackage[ngerman]{babel}			% neue deutsche Rechtschreibung
\usepackage[utf8]{inputenc} 		% Umlaute im Text
\usepackage[T1]{fontenc}
\usepackage{xspace}                 % Vermeidung von "ineinanderfallenden f's", wie z.B. bei Schifffahrt
\usepackage{url}		            % korrekte Anzeige/Umbruch von URLs
\usepackage{listings}               % z.B. nützlich zum Einbinden von Quellcode
\usepackage{hyperref} 				% für Hyperlinks in PDF-Dokumenten 
\usepackage{lmodern}
\usepackage{enumerate}
\usepackage{csquotes}
\usepackage{tabularray}
\usepackage{xcolor}
\usepackage{caption}
\captionsetup[wrapfigure]{format=plain, justification=justified, singlelinecheck=false}

\usepackage{wrapfig}

\usepackage{tocbibind}

\usepackage{subcaption}

\definecolor{bg}{rgb}{0.95,0.95,0.95}

\usepackage{minted}
\setminted{
	fontsize=\footnotesize,
	linenos,
	bgcolor=bg,
}
\usepackage{tcolorbox}

% Kopfzeile
\usepackage[headsepline,manualmark]{scrlayer-scrpage}
\clearpairofpagestyles
\ohead{\pagemark}
\ihead{\headmark}
\automark{chapter}
\pagestyle{scrheadings}

% Seitenspiegel
\usepackage[left=25mm,right=20mm,top=25mm,bottom=25mm]{geometry}

% Literatur
\usepackage[backend=biber, %% Hilfsprogramm "biber" (statt "biblatex" oder "bibtex")
            style=numeric, %% Zitierstil (siehe Dokumentation, bitte mit Betreuer absprechen)
            natbib=true, %% Bereitstellen von natbib-kompatiblen Zitierkommandos
            hyperref=true, %% hyperref-Paket verwenden, um Links zu erstellen
]{biblatex}

% Einbindung der Literatur-Datenbank
\addbibresource{./Literatur/quellen.bib}

% Grafiken
\usepackage{graphicx} 				% Grafiken einfügen (pdf,png - aber jpg vermeiden)
\graphicspath{{./Bilder/}}          % Pfad zu den Bildern
\usepackage{lipsum}                 % For placeholder text, you can remove this and add your text

% Tabellen
\usepackage{booktabs} 				% bessere Gestaltung von Tabellen
\usepackage{longtable} 				% für bessere Tabellen über mehrere Seiten

% Koma-Script Kompatibilität
\usepackage{scrhack}

% 1.5 Zeilenabstand
\usepackage{setspace}
\onehalfspacing

% Das Dokument selbst mit seinen Bestandteilen
% --------------------------------------------
\begin{document}
\frontmatter 
    % Titelseite soll keine Kopf oder Fußzeile haben
\thispagestyle{empty}



\vspace*{-20mm}
\begin{flushright}
\includegraphics[width=0.5\textwidth]{Bilder/LogoHS.png}
\end{flushright}


\vspace*{2cm}

% Alle Elemente sollen zentriert sein
\begin{center}
% Art der Arbeit => (Bachelorarbeit , Masterarbeit, Studienarbeit)
{\Large \textbf{Bachelorarbeit}}\\ 

\vspace*{1cm}

{\large Studiengang Informatik\\[1mm]}

\vspace{1cm}

% Titel der Arbeit 
{\Large \bfseries Titel der\\ 
	Arbeit\\
	auf drei Zeilen\\}


\vspace{1.5cm}

% Name des/der Autors/Autoren
{\large Vorname Nachname}\\[40mm]

\end{center}

\vfill

% Aufgabensteller, Kontaktdaten und Abgabetermin
\parbox{120mm}{
\begin{tabbing}
Aufgabensteller/Prüfer \hspace{.7cm} \= Prof. Dr. M. Genius\\
Arbeit vorgelegt am                  \> 1. April 2017\\
durchgeführt in der                  \> Fakultät Informatik\\[4mm]
% falls der praktische Teil der Arbeit in einer Firma durchgeführt wurde.
durchgeführt bei                     \> Fa. VeriBest GmbH, 12345 Stadt, Bereich ABC\\
% Die Nennung des Betreuers ist freiwillig und mit diesem abzustimmen
Betreuer                             \> Dipl.-Inf. Fred Feuerstein, Abt. DEF\\
                                     \> Tel/Email\\[4mm]
Anschrift des Verfassers             \> Straße Nummer\\
                                     \> PLZ Ort\\
Kontakt des Verfassers               \> mymail@mail.de\\
\end{tabbing}
}

 				    % Titelblatt
%    \newpage
\thispagestyle{empty}

% Bitte hier keine Änderungen vornehmen, sondern vollständig handschriftlich ausfüllen

\noindent  {\Large \textbf{Sperrvermerk}}\\ 

\vspace*{2cm}
\bfseries
\noindent Die nachfolgende Arbeit enthält vertrauliche Informationen und Daten der Firma
VeryBest GmbH. 

\medskip
\noindent Veröffentlichungen oder Vervielfältigungen - auch nur auszugsweise oder in
elektronischer Form - sind ohne ausdrückliche schriftliche Genehmigung der
Firma VeryBest GmbH nicht gestattet.
\medskip

\noindent Die Sperrfrist gilt bis zum 31. Dezember 20XY.

\medskip
\noindent Die Arbeit darf bis zum Ablauf der Sperrfrist nur für Prüfungszwecke verwendet
werden.
\normalfont 			% Sperrvermerk (nur falls von Firma verlangt)
    \include{./Bestandteile/kurzzusammenfassung}  	% Abstract
    \tableofcontents 					            % Inhaltsverzeichnis
    \clearpage
    \listoffigures  					 	        % Abbildungsverzeichnis
    \clearpage
    \listoftables						            % Tabellenverzeichnis 
    \clearpage
% ----------------------------------------------
\clearpairofpagestyles % Leere bestehende Kopf- und Fußzeilen
\cfoot*{\pagemark} % Seitenzahl unten in der Mitte
\mainmatter 						% die einzelnen Kapitel, bei Bedarf weitere *.tex Dateien erzeugen und hier einbinden
    \chapter{Einleitung}
In dieser Arbeit soll genauer untersucht werden, was Refactoring tatsächlich an Code verändert, welche Möglichkeiten es gibt ein bestimmtes Verhalten überhaupt darzustellen.
Als Grundlage dieser Arbeit gilt das 11. Kapitel aus dem Buch \textit{„five lines of code“} von Christian Clausen \cite{fiveLines.2023}.
Dieses wurde im Rahmen des Seminars „Refactoring“, bei Professor Dr. Georg Hagel, untersucht und für diese Arbeit aufgearbeitet. 
Das Kapitel des Buchs kann zwar eigenständig gelesen werden, aber ein grundlegendes Verständnis von Refactoring ist trotzdem erforderlich.\\
Außerdem wird erläutert, in welchen Situationen auf ein Refactoring verzichtet werden sollte und welche Gründe es dafür gibt.\\
Anschließend sollen einige Maßnahmen vorgestellt werden, mit denen Sicherheit erlangt werden kann, dass Code ordnungsgemäß funktioniert.
Gerade nach einem umfassenden Refactoring spielt dies eine große Rolle. 
Abschließend werden einige Fälle vorgestellt, in denen Refactoring aus verschiedenen Gründen häufig nicht durchgeführt wird und es wird gezeigt wieso dies der Fall sein sollte.
\chapter{Strukturen in der Softwareentwicklung}
Bevor man sich mit den Strukturen in der Softwareentwicklung auseinandersetzen kann, ist es wichtig, sich erneut vor Augen zu führen, was Software eigentlich ist.\\
"\emph{Software modelliert einen Teil der Welt. Die Welt - und unser Verständnis davon - entwickelt sich, und unsere Software muss sich entwickeln, um ein akkurates Modell zu sein.}" \citep[S. 311]{fiveLines.2023}\\
Dies heißt außerdem das Software nie fertig ist, da sie sich immer an die Ständig ändernde Welt anpassen.\\
Code bildet also verschiedenste Gegebenheiten aus der Realität ab. Darunter zählen Informationen, Zusammenhänge und ganze Abläufe.
Zusammen ergibt sich dadurch eine Struktur, ein wiedererkennbares Muster, welches sich sowohl in der echten Welt als auch in der Software finden lässt. \citep[S. 311]{fiveLines.2023}
\subsubsection{Verschiedene Arten von Strukturen}

Es gibt verschiedene Bereiche in der Softwareentwicklung, in denen Struktur eine Rolle spielt.
Es ist möglich diese an zwei Achsen einzuteilen.
Zum einen gibt es einige Faktoren welche den Menschen, also die Softwareentwickler direkt betreffen, oder aber den tatsächliche Code.\\
Auf der zweiten Achse wählt Clausen den Wirkungsbereich als Einteilung \citep[S. 311]{fiveLines.2023}.
Die folgende Tabelle zeigt das Ergebnis der Einteilung.


\begin{table} [ht]
	\centering
	\begin{tblr}{
		vline{2} = {-}{},
		hline{2} = {-}{},
	}
			& Teamintern           & Teamübergreifend           \\
	Code     & Daten und Funktionen & externe APIs               \\
	Menschen & Hierarchie, Prozesse & Verhalten, Domänenexperten 
	\end{tblr}
	\caption{Strukturkategorien \cite{fiveLines.2023} (korrigierte Form)}
	\label{tab:Auswertungskategorien}
\end{table}


Melvin E. Conway stellt bereits 1968 Beobachtungen an, dass es eine gewisse Symmetrie zwischen der Arbeitsweise von Entwicklerteams und den Zusammenhängen der tatsächlichen Systeme gibt. \cite{conway.1968}
\par
Auch das Nutzerverhalten kann bestimmte Strukturen vorgeben, da diese ein immer gleiches Verhalten erwarten.
Auch wenn einige Abläufe optimiert oder umstrukturiert werden können, ist es nicht immer sinnvoll dies zu tun, da dann ggf. Nutzer neu geschult werden müssen.\citep[S. 312]{fiveLines.2023}

	\chapter{Arten, wie Code Verhalten spiegelt}
Im nächsten Teil wird beleuchtet, wie Verhalten im Code abgebildet werden kann. Dabei gibt es grundlegend drei verschiedene Arten. Diese werden im folgenden anhand eines einfachen Beispiels erläutert. Des Weiteren soll auf die Erzeugung von Endlosschleifen eingegangen werden, wie Clausen feststellt, einen Sonderfall darstellen. \cite{fiveLines.2023}
\begin{tcolorbox}[colback=gray!20!white, colframe=gray!75!black, title=Beispielverhalten]
    Es soll bis zu einer bestimmten ganzen Zahl abwechselnd „gerade“ und „ungerade“ in der Konsole ausgegeben werden. „0“ wird hierbei als gerade angesehen.\\
    Dieses Verhalten ist am von Clausen genutzten Beispiel (FizzBuzz) nachempfunden und so weit wie möglich vereinfacht, um weiterhin alle nötigen Besonderheiten zu veranschaulichen.\cite{fiveLines.2023}
\end{tcolorbox}

\section{Verhalten im Kontrollfluss}
Die erste und wohl einfachste Möglichkeit, Verhalten im Code abzubilden, ist der Kontrollfluss. Dieser zeichnet sich durch die Verwendung von Kontrolloperatoren, Methodenaufrufen und der Zeilenabfolge aus. \cite{fiveLines.2023}
In folgender Abbildung werden dafür jeweils einfache Beispiele gezeigt. (\ref{fig:Kontrollfluss})
\begin{figure}[ht]
    \begin{subfigure}[t]{0.30\textwidth}
        \centering
        \begin{minipage}[t]{\linewidth}
            \begin{minted}[linenos=false]{typescript}
const i = 0;
while (i < 5) {
    foo(i);
    i++;
}
            \end{minted}
        \end{minipage}
        \caption{Kontrolloperatoren}
        \label{fig:Kontrolloperatoren}
    \end{subfigure}
    \hfill
    \begin{subfigure}[t]{0.30\textwidth}
        \centering
        \begin{minipage}[t]{\linewidth}
            \begin{minted}[linenos=false]{typescript}
function loop(i: number) {
    if (i < 5) {
        foo(i);
        loop(i + 1);
    }
}
            \end{minted}
        \end{minipage}
        \caption{Methodenaufrufe}
        \label{fig:Methodenaufrufe}
    \end{subfigure}
    \hfill
    \begin{subfigure}[t]{0.30\textwidth}
        \centering
        \begin{minipage}[t]{\linewidth}
            \begin{minted}[linenos=false]{typescript}
foo(0);
foo(1);
foo(2);
foo(3);
foo(4);
            \end{minted}
        \end{minipage}
        \caption{Zeilenabfolge}
        \label{fig:Zeilenabfolge}
    \end{subfigure}
    \caption{Beispiele für Code im Kontrollfluss \cite{fiveLines.2023}}
    \label{fig:Kontrollfluss}
\end{figure}

Der Unterschied dieser Unterkategorien wird bei der Betrachtung des Aufrufs „foo(i)“ und dessen Werthereingabe deutlich. Bei \ref{fig:Kontrolloperatoren} wird mithilfe des „while“ Operators die Funktion aufgerufen und die Eingabe erhöht.\\ Das mittlere Beispiel zeigt die Verwendung einer rekursiven Methode, um das Verhalten darzustellen. Der Eingabeparameter dient hier als Wert für den Funktionsaufruf.\\ Das letzte Beispiel \ref{fig:Zeilenabfolge} zeigt das gleiche Verhalten durch einfache Aufrufe. Hier wird die Funktion mit Wert im Klartext aufgerufen.

\subsection{Eigenes Beispiel}
Folgende Abbildung zeigt das Beispiel-Verhalten im Kontrollfluss.
\begin{figure}[h]
    \centering
        \begin{minted}{typescript}
function istGerade(n: number) {
    for(let i = 0; i <= n; i++) {
        if(i % 2 == 0) {
            console.log("Gerade");
        } else {
            console.log("Ungerade");
        }
    }
}
        \end{minted}
    \caption{Beispiel im Kontrollfluss}
    \label{fig:KontrollflussIstGerade}
\end{figure}\\
Um die Unterschiede der verschiedenen Darstellungsformen zu erkennen, ist es sinnvoll die Aufrufe von \textit{„console.log()“} zu betrachten. Diese stellen bei unserem Beispiel die tatsächlich durchzuführende Aktion dar. In Abbildung \ref{fig:KontrollflussIstGerade} lässt sich erkennen, dass diese Aufrufe durch die Kontrolloperatoren \textit{for} und \textit{if} gesteuert werden.
\subsection{Vor und Nachteile}
Da das Programmieren im Kontrollfluss schnell und einfach funktioniert, eignet sich diese Art gut, um neues Vehalten initial abzubilden. 
\section{Verhalten in der Struktur der Daten}
\subsection{Eigenes Beispiel}
\begin{figure}[ht]
    \centering
        \begin{minted}{typescript}
interface Zahl{
    istGerade(): void;
}

class GeradeZahl implements Zahl{
    constructor(private count: number) {}
    istGerade() {
        console.log("Gerade");
        if(this.count != 0) {
            new UngeradeZahl(this.count - 1).istGerade();
        }
    }
}

class UngeradeZahl implements Zahl{
    constructor(private count: number) {}
    istGerade() {
        console.log("Ungerade");
        if(this.count != 0) {
            new GeradeZahl(this.count - 1).istGerade();
        }
    }
}
        \end{minted}
    \caption{Beispiel in einer Datenstruktur}
    \label{fig:KontrollflussIstGerade}
\end{figure}

\section{Verhalten in den Daten}
\subsection{Eigenes Beispiel}
\begin{figure}[ht]
    \centering
        \begin{minted}{typescript}
const daten: (() => void)[] = [
    () => console.log("Gerade"),
    () => console.log("Ungerade")
];

function istGerade(n: number) {
    for(let i = 0; i <= n; i++) {
        daten[i % daten.length]();
    }
}
        \end{minted}
    \caption{Beispiel in einer Datenstruktur}
    \label{fig:KontrollflussIstGerade}
\end{figure}
    \chapter{Ist Refactoring immer sinnvoll?}
\section{Beobachten statt vorhersagen}


\chapter{Sicherheit gewinnen, ohne den Code zu verstehen}
\section{Sicherheit durch Tests}
Die wohl einfachste Möglichkeit die Korrektheit von Software zu überprüfen, ist das Testen. In der Softwareentwicklung gibt es verschiedene Arten, wie richtiges Verhalten, einzelne Codeabschnitte oder ganze Produkte getestet werden können. Einige dieser Arten werden im folgenden aufgelistet.
\begin{itemize}
  \item \textbf{Unit Tests:} Testen von bla
  \item \textbf{Whiteboxtests:} Testen von bla
\end{itemize}
\section{Sicherheit durch Handwerkskunst}
\section{Sicherheit durch Werkzeuge}
\lipsum[1] % Dummy text

\begin{wrapfigure}{r}{0.4\textwidth}
  \centering
  \includegraphics[width=0.35\textwidth]{Bilder/screenshotWebstorm} % Replace with your image file
  \caption{Screenshot WebStorm Refactoring-Tools \cite{webstorm.2024}}
\end{wrapfigure}

\lipsum[2-4] % More dummy text
\section{Sicherheit durch formale Verifikation}
\section{Sicherheit durch Fehlertoleranz}
% ----------------------------------------------
\backmatter 

\printbibliography[heading=bibintoc]

\input{./Bestandteile/erklaerung} 	% Erklärungen - Unterschreiben nicht vergessen!

\end{document}

% Vorlage erstellt von Alexander Bartel (alexander.bartel@hs-kempten.de), Fakultät Informatik, Hochschule Kempten (c) CC BY 4.0
% ergänzt durch: Prof. Dr. Rieck (stefan.rieck@hs-kempten.de)
% 20230329 SR umgestellt auf biblatex und biber
