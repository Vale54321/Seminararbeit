\documentclass[12pt,oneside,listof=totoc,paper=a4,headings=small]{scrbook}
% Nützliche Packages für die Gestaltung und allgemeine Konfiguration des Dokuments
% -----------------------------------

% Allgemeine Formatierungen
\usepackage[ngerman]{babel}			% neue deutsche Rechtschreibung
\usepackage[utf8]{inputenc} 		% Umlaute im Text
\usepackage[T1]{fontenc}
\usepackage{xspace}                 % Vermeidung von "ineinanderfallenden f's", wie z.B. bei Schifffahrt
\usepackage{url}		            % korrekte Anzeige/Umbruch von URLs
\usepackage{listings}               % z.B. nützlich zum Einbinden von Quellcode
\usepackage{hyperref} 				% für Hyperlinks in PDF-Dokumenten 
\usepackage{lmodern}
\usepackage{enumerate}
\usepackage{csquotes}
\usepackage{tabularray}
\usepackage{listings}
\usepackage{xcolor}

% Kopfzeile
\usepackage[headsepline,manualmark]{scrlayer-scrpage}
\clearpairofpagestyles
\ohead{\pagemark}
\ihead{\headmark}
\automark{chapter}
\pagestyle{scrheadings}

% Seitenspiegel
\usepackage[left=25mm,right=20mm,top=25mm,bottom=25mm]{geometry}

\definecolor{codegreen}{rgb}{0,0.6,0}
\definecolor{codegray}{rgb}{0.5,0.5,0.5}
\definecolor{codepurple}{rgb}{0.58,0,0.82}
\definecolor{codeorange}{rgb}{1,0.5,0}
\definecolor{backcolour}{rgb}{0.95,0.95,0.92}

\renewcommand{\lstlistingname}{Code-Beispiel}

\lstdefinelanguage{TypeScript}{
	sensitive=true,
	morecomment=[l]{//},
	morecomment=[s]{/*}{*/},
	morestring=[b]",
	keywords=[1]{let, const, break, case, catch, class, const, continue, debugger, default, delete, do, else, enum, export, extends, finally, for, function, if, import, in, instanceof, new, return, super, switch, this, throw, try, typeof, var, void, while, with},
	keywordstyle=[1]\color{blue},
	keywords=[2]{true, false, null,console},
	keywordstyle=[2]\color{codepurple},
	keywords=[3]{string, number, boolean, any, void},
	keywordstyle=[3]\color{codeorange}\bfseries,
	identifierstyle=\color{black},
	commentstyle=\color{gray}\textit,
	stringstyle=\color{codegreen},
	morestring=[b]',
	morestring=[b]`,
}


\lstdefinestyle{mystyle}{
	backgroundcolor=\color{backcolour},   
	commentstyle=\color{codegreen},
	keywordstyle=\color{blue},
	numberstyle=\tiny\color{codegray},
	stringstyle=\color{codepurple},
	basicstyle=\ttfamily\footnotesize,
	breakatwhitespace=false,         
	breaklines=true,                 
	captionpos=b,                    
	keepspaces=true,                 
	numbers=left,                    
	numbersep=5pt,                  
	showspaces=false,                
	showstringspaces=false,
	showtabs=false,                  
	tabsize=2
	}

\lstset{style=mystyle}

% Literatur
\usepackage[backend=biber, %% Hilfsprogramm "biber" (statt "biblatex" oder "bibtex")
            style=numeric, %% Zitierstil (siehe Dokumentation, bitte mit Betreuer absprechen)
            natbib=true, %% Bereitstellen von natbib-kompatiblen Zitierkommandos
            hyperref=true, %% hyperref-Paket verwenden, um Links zu erstellen
]{biblatex}

% Einbindung der Literatur-Datenbank
\addbibresource{./Literatur/quellen.bib}

% Grafiken
\usepackage{graphicx} 				% Grafiken einfügen (pdf,png - aber jpg vermeiden)
\graphicspath{{./Bilder/}}          % Pfad zu den Bildern

% Tabellen
\usepackage{booktabs} 				% bessere Gestaltung von Tabellen
\usepackage{longtable} 				% für bessere Tabellen über mehrere Seiten

% Koma-Script Kompatibilität
\usepackage{scrhack}



% Das Dokument selbst mit seinen Bestandteilen
% --------------------------------------------
\begin{document}
\frontmatter 
    % Titelseite soll keine Kopf oder Fußzeile haben
\thispagestyle{empty}



\vspace*{-20mm}
\begin{flushright}
\includegraphics[width=0.5\textwidth]{Bilder/LogoHS.png}
\end{flushright}


\vspace*{2cm}

% Alle Elemente sollen zentriert sein
\begin{center}
% Art der Arbeit => (Bachelorarbeit , Masterarbeit, Studienarbeit)
{\Large \textbf{Bachelorarbeit}}\\ 

\vspace*{1cm}

{\large Studiengang Informatik\\[1mm]}

\vspace{1cm}

% Titel der Arbeit 
{\Large \bfseries Titel der\\ 
	Arbeit\\
	auf drei Zeilen\\}


\vspace{1.5cm}

% Name des/der Autors/Autoren
{\large Vorname Nachname}\\[40mm]

\end{center}

\vfill

% Aufgabensteller, Kontaktdaten und Abgabetermin
\parbox{120mm}{
\begin{tabbing}
Aufgabensteller/Prüfer \hspace{.7cm} \= Prof. Dr. M. Genius\\
Arbeit vorgelegt am                  \> 1. April 2017\\
durchgeführt in der                  \> Fakultät Informatik\\[4mm]
% falls der praktische Teil der Arbeit in einer Firma durchgeführt wurde.
durchgeführt bei                     \> Fa. VeriBest GmbH, 12345 Stadt, Bereich ABC\\
% Die Nennung des Betreuers ist freiwillig und mit diesem abzustimmen
Betreuer                             \> Dipl.-Inf. Fred Feuerstein, Abt. DEF\\
                                     \> Tel/Email\\[4mm]
Anschrift des Verfassers             \> Straße Nummer\\
                                     \> PLZ Ort\\
Kontakt des Verfassers               \> mymail@mail.de\\
\end{tabbing}
}

 				    % Titelblatt
%    \newpage
\thispagestyle{empty}

% Bitte hier keine Änderungen vornehmen, sondern vollständig handschriftlich ausfüllen

\noindent  {\Large \textbf{Sperrvermerk}}\\ 

\vspace*{2cm}
\bfseries
\noindent Die nachfolgende Arbeit enthält vertrauliche Informationen und Daten der Firma
VeryBest GmbH. 

\medskip
\noindent Veröffentlichungen oder Vervielfältigungen - auch nur auszugsweise oder in
elektronischer Form - sind ohne ausdrückliche schriftliche Genehmigung der
Firma VeryBest GmbH nicht gestattet.
\medskip

\noindent Die Sperrfrist gilt bis zum 31. Dezember 20XY.

\medskip
\noindent Die Arbeit darf bis zum Ablauf der Sperrfrist nur für Prüfungszwecke verwendet
werden.
\normalfont 			% Sperrvermerk (nur falls von Firma verlangt)
    \include{./Bestandteile/kurzzusammenfassung}  	% Abstract
    \tableofcontents 					            % Inhaltsverzeichnis
    \clearpage
    \listoffigures  					 	        % Abbildungsverzeichnis
    \clearpage
    \listoftables						            % Tabellenverzeichnis 
    \clearpage
% ----------------------------------------------
\mainmatter 						% die einzelnen Kapitel, bei Bedarf weitere *.tex Dateien erzeugen und hier einbinden
    \chapter{Strukturen in der Softwareentwicklung}

Software modelliert Teile der Welt. Informationen, Zusammenhänge und Abläufe werden abgebildet.


\section{Verschiedene Arten von Strukturen}
Es ist möglich die Softwareentwicklung an zwei Achsen zu trennen. Dadurch ergibt sich die in der folgenden Tabelle gezeigte Einteilung.
\begin{table} [ht]
	\centering
	\begin{tblr}{
		vline{2} = {-}{},
		hline{2} = {-}{},
	}
			& Teamintern           & Teamübergreifend           \\
	Code     & Daten und Funktionen & Externe APIs               \\
	Menschen & Hierarchie, Prozesse & Verhalten, Domänenexperten 
	\end{tblr}
	\caption{Strukturkategorien}
	\label{tab:Auswertungskategorien}
\end{table}

\section{Weitere Einschränkungen}

Auch das Nutzerverhalten kann bestimmte Strukturen vorgeben, da diese ein immer gleiches Verhalten erwarten. Auch wenn einige Abläufe optimiert oder umstrukturiert werden können, ist es nicht immer sinnvoll dies zu tun, da dann ggf. Nutzer neu geschult werden müssen.\cite{fiveLines.2023}

% ----------------------------------------------
\backmatter 

\printbibliography

\input{./Bestandteile/erklaerung} 	% Erklärungen - Unterschreiben nicht vergessen!

\end{document}

% Vorlage erstellt von Alexander Bartel (alexander.bartel@hs-kempten.de), Fakultät Informatik, Hochschule Kempten (c) CC BY 4.0
% ergänzt durch: Prof. Dr. Rieck (stefan.rieck@hs-kempten.de)
% 20230329 SR umgestellt auf biblatex und biber
