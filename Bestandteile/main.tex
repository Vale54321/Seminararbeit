\chapter{Strukturen in der Softwareentwicklung}

Software modelliert Teile der Welt. Informationen, Zusammenhänge und Abläufe werden abgebildet.


\section{Verschiedene Arten von Strukturen}
Es ist möglich die Softwareentwicklung an zwei Achsen zu trennen. Dadurch ergibt sich die in der folgenden Tabelle gezeigte Einteilung.
\begin{table} [ht]
	\centering
	\begin{tblr}{
		vline{2} = {-}{},
		hline{2} = {-}{},
	}
			& Teamintern           & Teamübergreifend           \\
	Code     & Daten und Funktionen & Externe APIs               \\
	Menschen & Hierarchie, Prozesse & Verhalten, Domänenexperten 
	\end{tblr}
	\caption{Strukturkategorien}
	\label{tab:Auswertungskategorien}
\end{table}

\section{Weitere Einschränkungen}

Auch das Nutzerverhalten kann bestimmte Strukturen vorgeben, da diese ein immer gleiches Verhalten erwarten. Auch wenn einige Abläufe optimiert oder umstrukturiert werden können, ist es nicht immer sinnvoll dies zu tun, da dann ggf. Nutzer neu geschult werden müssen.\cite{fiveLines.2023}

\chapter{Arten, wie Code Verhalten spiegelt}
\section{Verhalten im Kontrollfluss}
Die erste und wohl einfachste Möglichkeit, Verhalten im Code abzubilden, ist der Kontrollfluss. Dieser zeichnet sich durch die Verwendung von Kontrolloperatoren, Methodenaufrufen und der Zeilenabfolge aus.\cite{wikibook}

\begin{lstlisting}[language=TypeScript, caption={TypeScript-Code}, label=lst:typescript]
	function helloWorld() {
		console.log("Hallo, Welt!");
	}
	
	helloWorld();
	let y: number = 2;
	
	y = y + 1;
	console.log(y);
\end{lstlisting}